\documentclass{article}

\usepackage{amsmath}

\usepackage{enumitem}
\setlist[itemize,enumerate]{noitemsep}

\newtheorem{definition}{Definition}
\newcommand{\E}{\mathcal{E}}
\newcommand{\KG}{\mathcal{KG}}
\newcommand{\R}{\mathcal{R}}
\newcommand{\obs}{{<t}}
\newcommand{\related}{\text{related}}
\newcommand{\new}{\text{new}}
\newcommand{\noent}{\emptyset}

\title{
  Notes on {\it Barack's wife Hilary: Using Knowledge Graphs for Fact-Aware
  Language Modeling}
}
\author{
  Matthew Feng
}

\begin{document}
\maketitle

\section{Article info}
\paragraph{Authors} Robert L. Logan IV, Nelson F. Liu, Matthew E. Peters,
Matt Gardner, Sameer Singh

\section{Notes}

\subsection{Big ideas}
\begin{itemize}
  \item KGLM has mechanisms for {\it selecting and copying facts} from
    a knowledge graph that are relevant to the context.
  \item Introduce the {\bf Linked WikiText-2 dataset}
  \item The KGLM has a {\it dynamically growing local knowledge graph},
    a subset of the actual knowledge graph.
\end{itemize}

\subsection{Additional readings}
\begin{enumerate}
  \item AWD-LSTM (Merity et al. 2018)
  \item
\end{enumerate}

\subsection{Details}

\begin{definition}
  A {\bf knowledge graph} is a directed, labeled graph with entities
  $\E$ as nodes, and relations $\R$ as edge labels.
  $\KG$ = $\{(p, r, e) | p \in \E, r \in \R, e \in \E\}$.

  Practical knowledge graphs have other characteristics:
  \begin{itemize}
    \item some relations are literal values
    \item facts may be expressed as properties on relations
    \item entities have {\it aliases}
  \end{itemize}
\end{definition}


\paragraph{Goal} Compute $p(x_t, \E_t | x_\obs, \E_\obs)$

\subsubsection{Generative algorithm}
\begin{enumerate}
  \item Decide the type of $x_t$, denoted $t_t$, where
    $t_t \in \{\related, \new, \noent\}$.
\end{enumerate}

\end{document}
